%!TEX root = ts.tex

\rSec0[dcl.dcl]{Declarations}

\rSec1[dcl.spec]{Specifiers}

\setcounter{subsection}{6}
\rSec2[dcl.type]{Type specifiers}

\setcounter{subsubsection}{1}
\rSec3[dcl.type.simple]{Simple type specifiers}

\pnum
In C++ [dcl.type.simple], apply the following change:

\begin{std.txt}
The simple type specifiers are

\begin{bnf}
\nontermdef{simple-type-specifier}\br
    \opt{nested-name-specifier} type-name\br
    nested-name-specifier \terminal{template} simple-template-id\br
    \opt{nested-name-specifier} template-name\br
    \terminal{char}\br
    \terminal{char16_t}\br
    \terminal{char32_t}\br
    \terminal{wchar_t}\br
    \terminal{bool}\br
    \terminal{short}\br
    \terminal{int}\br
    \terminal{long}\br
    \terminal{signed}\br
    \terminal{unsigned}\br
    \terminal{float}\br
    \terminal{double}\br
    \terminal{void}\br
    \terminal{auto}\br
    decltype-specifier\br
    \added{reflexpr-specifier}
\end{bnf}

\begin{bnf}
\nontermdef{type-name}\br
    class-name\br
    enum-name\br
    typedef-name\br
    simple-template-id
\end{bnf}

\begin{bnf}
\nontermdef{decltype-specifier}\br
  \terminal{decltype} \terminal{(} expression \terminal{)}\br
  \terminal{decltype} \terminal{(} \terminal{auto} \terminal{)}
\end{bnf}

\begin{bnf}
\added{\nontermdef{reflexpr-specifier}}\br
  \added{\terminal{reflexpr} \terminal{(} reflexpr-operand \terminal{)}\br}
\end{bnf}

\begin{bnf}
\added{\nontermdef{reflexpr-operand}}\br
  \added{\terminal{::}}\br
  \added{type-id}\br
  \added{\opt{nested-name-specifier} namespace-name}\br
  \added{id-expression}\br
  \added{\terminal{(} expression \terminal{)}}\br
  \added{function-call-expression}\br
  \added{functional-type-conv-expression}\br
\end{bnf}

...

The other \grammarterm{simple-type-specifiers} specify either a
previously-declared type, a type determined from an expression, \added{a
reflection meta-object type (\ref{dcl.type.reflexpr})}, or one of the
fundamental types (\ref{basic.fundamental}).
\end{std.txt}

\pnum
Append the following row to Table 11:

\begin{center}
\begin{TableBase}
\begin{tabular}{|ll|}
\topline
\added{\tcode{reflexpr} \tcode{(} \grammarterm{reflexpr-operand} \tcode{)}\br} &
\added{the type as defined below} \\
\bottomline
\end{tabular}
\end{TableBase}
\end{center}

\pnum
At the end of \cxxref{dcl.type.simple}, insert the following paragraph:

\begin{std.txt}
\added{For a \grammarterm{reflexpr-operand} \tcode{x}, the type denoted
by \tcode{reflexpr(x)} is a type that satisfies
the constraints laid out in \ref{dcl.type.reflexpr}.}
\end{std.txt}


\setcounter{subsubsection}{5}
\rSec3[dcl.type.reflexpr]{Reflection type specifiers}

Insert the following subclause:

\begin{std.txt}\color{addclr}

    \pnum
    The \grammarterm{reflexpr-specifier} yields a type \tcode{T} that allows
    inspection of some properties of its operand through type traits or type
    transformations on \tcode{T} (\ref{reflect.ops}).  The operand to the
    \grammarterm{reflexpr-specifier} shall be a type, namespace, enumerator,
    variable, structured binding, data member, function parameter, captured
    entity, parenthesized expression, \grammarterm{function-call-expression} or
    \grammarterm{functional-type-conv-expression}.  Any such \tcode{T} satisfies the requirements of
    \tcode{reflect::Object} (\ref{reflect.concepts}) and other \tcode{reflect}
    concepts, depending on the operand.  A type satisfying the requirements of
    \tcode{reflect::Object} is called a \emph{meta-object type}.  A meta-object
    type is an unnamed, incomplete namespace-scope class type (Clause \cxxref{class}).

    \pnum
    An entity or alias \tcode{B} is \emph{reflection-related} to an entity or
    alias \tcode{A} if

    \begin{itemize}
      \item \tcode{A} and \tcode{B} are the same entity or alias,
      \item \tcode{A} is a variable or enumerator and \tcode{B} is the type of \tcode{A},
      \item \tcode{A} is an enumeration and \tcode{B} is the underlying type of \tcode{A},
      \item \tcode{A} is a class and \tcode{B} is a member or base class of \tcode{A},
      \item \tcode{A} is a non-template alias that designates the entity \tcode{B},
      \item \tcode{A} is not the global namespace and \tcode{B} is an enclosing class or namespace of \tcode{A},
      \item \tcode{A} is the parenthesized expression \tcode{( B )},
      \item \tcode{A} is a lambda capture of the closure type \tcode{B},
      \item \tcode{A} is the closure type of the lambda capture \tcode{B},
      \item \tcode{B} is the type specified by the \grammarterm{functional-type-conv-expression} \tcode{A},
      \item \tcode{B} is the function selected by overload resolution for a \grammarterm{function-call-expression} \tcode{A},
      \item \tcode{B} is the return type, a parameter type, or function type of the function \tcode{A}, or
      \item \tcode{B} is reflection-related to an entity or alias \tcode{X} and \tcode{X} is reflection-related to \tcode{A}.
    \end{itemize}

\pnum
\begin{note}
This relationship is reflexive and transitive, but not symmetric.
\end{note}

\pnum
\begin{example}
\begin{codeblock}
struct X;
struct B  {
   using X = ::X;
   typedef X Y;
};
struct D : B {
   using B::Y;
};
\end{codeblock}

\pnum
The alias \tcode{D::Y} is reflection-related to \tcode{::X}, but not to \tcode{B::Y} or \tcode{B::X}.
\end{example}

\pnum
Zero or more successive applications of type transformations that yield
meta-object types (\ref{reflect.ops}) to the type denoted by a \emph{reflexpr-specifier}
enable inspection of entities and aliases that are reflection-related to the
operand; such a meta-object type is said to \emph{reflect} the respective
reflection-related entity or alias.

\pnum
\begin{example}
\begin{codeblock}
template <typename T> std::string get_type_name() {
   namespace reflect = std::experimental::reflect;
   // T_t is an Alias reflecting T:
   using T_t = reflexpr(T);
   // aliased_T_t is a Type reflecting the type for which T is a synonym:
   using aliased_T_t = reflect::get_aliased_t<T_t>;
   return reflect::get_name_v<aliased_T_t>;
}

std::cout << get_type_name<std::string>(); // outputs \tcode{basic_string}
\end{codeblock}
\end{example}

\pnum
It is unspecified whether repeatedly applying \tcode{reflexpr} to the same
operand yields the same type or a different type.
\begin{note}
If a meta-object type reflects an incomplete class type, certain type
transformations (\ref{reflect.ops}) cannot be applied.
\end{note}

\pnum
\begin{example}
\begin{codeblock}
class X;
using X1_m = reflexpr(X);
class X {};
using X2_m = reflexpr(X);
using X_bases_1 = std::experimental::reflect::get_base_classes_t<X1_m>; // OK:
                                             // X1_m reflects complete class X
using X_bases_2 = std::experimental::reflect::get_base_classes_t<X2_m>; // OK
std::experimental::reflect::get_reflected_type_t<X1_m> x; // OK: type X is complete
\end{codeblock}
\end{example}

\pnum
For the operand \tcode{::}, the type specified by the
\grammarterm{reflexpr-specifier} satisfies \tcode{reflect::GlobalScope}.

\pnum
The \grammarterm{identifier} or \grammarterm{simple-template-id} is looked up
using the rules for name lookup (\cxxref{basic.lookup}): if a
\grammarterm{nested-name-specifier} is included in the operand, qualified lookup
(\cxxref{basic.lookup.qual}) of \grammarterm{nested-name-specifier identifier} or
\grammarterm{nested-name-specifier simple-template-id} will be performed,
otherwise unqualified lookup (\cxxref{basic.lookup.unqual}) of
\grammarterm{identifier} or \grammarterm{simple-template-id} will be performed.
The type specified by the \grammarterm{reflexpr-specifier} satisfies concepts
depending on the result of name lookup, as shown in
Table~\ref{tab:reflect.concepts}.

\setcounter{table}{11}

\begin{floattable}{\added{\tcode{reflect} concept (\ref{reflect.concepts}) that the type specified by
a \emph{reflexpr-specifier} satisfies, for a given \emph{reflexpr-operand}.}}{tab:reflect.concepts}{l|l|l}
\topline
\added{\hdstyle{Category}} & \added{\hdstyle{\grammarterm{identifier} or \grammarterm{simple-template-id} kind }} & \added{\hdstyle{\tcode{reflect} Concept}}       \\ \capsep
\added{Type}        & \added{\grammarterm{class-name} designating a union}                           & \added{\tcode{reflect::Record}}         \\ \cline{2-3}
            & \added{\grammarterm{class-name} designating a closure type}                     & \added{\tcode{reflect::Lambda}}          \\ \cline{2-3}
            & \added{\grammarterm{class-name} designating a non-union class}                  & \added{\tcode{reflect::Class}}           \\ \cline{2-3}
            & \added{\grammarterm{enum-name}}                                                 & \added{\tcode{reflect::Enum}}            \\ \cline{2-3}
            & \added{template \grammarterm{type-parameter}}                           & \added{both \tcode{reflect::Type}}       \\
            &                                                                         & \added{and \tcode{reflect::Alias}}       \\ \cline{2-3}
            & \added{\grammarterm{decltype-specifier}}                                & \added{both \tcode{reflect::Type}}       \\
            &                                                                         & \added{and \tcode{reflect::Alias}}       \\ \cline{2-3}
            & \added{\grammarterm{type-name} introduced by a \grammarterm{using-declaration}} & \added{\tcode{reflect::Type},}       \\
            &                                                                                 & \added{\tcode{reflect::Alias}, and}  \\
            &                                                                                 & \added{\tcode{reflect::ScopeMember}} \\ \cline{2-3}
            & \added{any other \grammarterm{typedef-name}}                                    & \added{both \tcode{reflect::Type}}       \\
            &                                                                         & \added{and \tcode{reflect::Alias}}       \\ \cline{2-3}
            & \added{any other \grammarterm{type-id}}                                 & \added{\tcode{reflect::Type}}            \\ \cline{1-3}

\added{Namespace}   & \added{\grammarterm{namespace-alias}}                                           & \added{both \tcode{reflect::Namespace}}  \\
            &                                                                         & \added{and \tcode{reflect::Alias}}       \\ \cline{2-3}
            & \added{any other \grammarterm{namespace-name}}                          & \added{\tcode{reflect::Namespace}}       \\ \cline{1-3}
\added{Expression} & \added{the name of a data member}                                               & \added{\tcode{reflect::Variable}} \\ \cline{2-3}
            & \added{the name of a variable or structured binding}                            & \added{\tcode{reflect::Variable}}        \\ \cline{2-3}
            & \added{the name of an enumerator}                                               & \added{\tcode{reflect::Enumerator}}      \\ \cline{2-3}
            & \added{the name of a function parameter}                                        & \added{\tcode{reflect::FunctionParameter}} \\ \cline{2-3}
            & \added{the name of a captured entity}                                           & \added{\tcode{reflect::LambdaCapture}} \\ \cline{2-3}
            & \added{parenthesized expression; see Note A, below}                             & \added{\tcode{reflect::ParenthesizedExpression}} \\ \cline{2-3}
            & \added{\grammarterm{function-call-expression}; see Note B, below}               & \added{\tcode{reflect::FunctionCallExpression}} \\ \cline{2-3}
            & \added{\grammarterm{functional-type-conv-expression}; see Note C, below}        & \added{\tcode{reflect::FunctionalTypeConversion}} \\
\bottomline
\end{floattable}

\pnum
Note A: For a \grammarterm{reflexpr-operand} that is a parenthesized expression \tcode{(E)}, \tcode{E} shall be a \grammarterm{function-call-expression}, \grammarterm{functional-type-conv-expression}, or an expression \tcode{(E')} that satisfies the requirements for being a \grammarterm{reflexpr-operand}.

\pnum
Note B: If the \grammarterm{postfix-expression} of the \grammarterm{function-call-expression} is of class type, the function call shall not resolve to a surrogate call function (\cxxref{over.call.object}). Otherwise, the \grammarterm{postfix-expression} shall name a function that is the unique result of overload resolution.

\pnum
Note C: \begin{note}
The usual disambiguation between function-style cast and a \grammarterm{type-id} (\cxxref{dcl.ambig.res}) applies. \begin{example}The default constructor of class \tcode{X} can be reflected on as \tcode{reflexpr((X()))}, while \tcode{reflexpr(X())} reflects the type of a function returning \tcode{X}. \end{example}
\end{note}

\pnum
Any other \grammarterm{reflexpr-operand} renders the program ill-formed.

\pnum 
If the \grammarterm{reflexpr-operand} of the form \grammarterm{id-expression} is a constant expression, the type specified by the \grammarterm{reflexpr-specifier} also satisfies \tcode{reflect::Constant}.

\pnum
If the \grammarterm{reflexpr-operand} designates a name whose declaration is enclosed in a block
scope (\cxxref{basic.scope.block}) or function parameter scope
(\cxxref{basic.scope.param}) and the named entity is neither captured nor a function
parameter, the program is ill-formed.  If the \grammarterm{reflexpr-operand}
designates a class member, the type represented by the
\grammarterm{reflexpr-specifier} also satisfies \tcode{reflect::RecordMember}.
If the \grammarterm{reflexpr-operand} designates an expression, it is an
unevaluated operand (\cxxref{expr.context}).  If the
\grammarterm{reflexpr-operand} designates both an alias and a class name, the
type represented by the \grammarterm{reflexpr-specifier} reflects the alias and
satisfies \tcode{reflect::Alias}.


\end{std.txt}
