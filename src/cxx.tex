% Define section from the C++ standard that can be indexed
% using its dotted identifer. That is:
%
%  \cxxsec{basic.def.odr}{3.2}
%
% This is used to make references to sections of the C++ Standard
% that are not labeled within this document.
\newcommand{\cxxsec}[2]{%
  \expandafter\def\csname #1 \endcsname{#2}%
}

% Generate a reference to the section with the given id. This
% expands to the full chapter/section/subsection number declared
% by \cxxsec. For example:
%
%  \cxxref{basic.def.odr}
%
% Expands to the string 3.2.
\newcommand{\cxxref}[1]{%
  \csname #1 \endcsname%
}

\cxxsec{lex.key}{2.12}
\cxxsec{tab:keywords}{4}


\cxxsec{basic}{3}
\cxxsec{basic.def.odr}{3.2}
\cxxsec{basic.def.odr.odrparagraph1}{3}
\cxxsec{basic.def.odr.odrparagraph2}{6}

\cxxsec{basic.scope.block}{3.3.3}
\cxxsec{basic.scope.temp}{3.3.9}
\cxxsec{basic.lookup}{3.4}
\cxxsec{basic.lookup.unqual}{3.4.1}
\cxxsec{basic.lookup.qual}{3.4.3}
\cxxsec{basic.stc.dynamic}{3.7.4}
\cxxsec{basic.fundamental}{3.9.1}
\cxxsec{basic.fundamental.paragraph1}{9}

\cxxsec{expr}{5}
\cxxsec{expr.prim.general}{5.1.1}
\cxxsec{expr.prim.lambda.paragraph1}{18}
\cxxsec{expr.call}{5.2.2}
\cxxsec{expr.type.conv}{5.2.3}
\cxxsec{expr.typeid}{5.2.8}
\cxxsec{expr.cond}{5.16}
\cxxsec{expr.comma}{5.18}
\cxxsec{expr.const}{5.19}
\cxxsec{stmt.expr}{6.2}
\cxxsec{stmt.return}{6.6.3}

\cxxsec{dcl.fct.spec}{7.1.2}
\cxxsec{dcl.type.simple}{7.1.6.2}
\cxxsec{tref:simple.type.specifiers}{10}
\cxxsec{dcl.enum}{7.2}
\cxxsec{namespace.udecl}{7.3.3}

\cxxsec{dcl.ambig.res}{8.2}
\cxxsec{dcl.fct}{8.3.5}

\cxxsec{class}{9}
\cxxsec{class.nest}{9.7}
\cxxsec{class.access.base}{11.2}
\cxxsec{special}{12}
\cxxsec{class.conv.fct}{12.3.2}
\cxxsec{over.call.object}{13.3.1.1.2}
\cxxsec{over.oper}{13.5}
\cxxsec{temp.concept}{17.6.8}
\cxxsec{temp.dep.expr}{14.6.2.2}
\cxxsec{temp.deduct}{14.8.2}
\cxxsec{cpp.predefined}{16.8}
\cxxsec{tab:cpp.library.headers}{14}
\cxxsec{tab:cpp.headers.freestanding}{16}
\cxxsec{meta.rqmts}{20.10.1}
