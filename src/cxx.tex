% Define section from the C++ standard that can be indexed
% using its dotted identifer. That is:
%
%  \cxxsec{basic.def.odr}{3.2}
%
% This is used to make references to sections of the C++ Standard
% that are not labeled within this document.
\newcommand{\cxxsec}[2]{%
  \expandafter\def\csname #1 \endcsname{#2}%
}

% Generate a reference to the section with the given id. This
% expands to the full chapter/section/subsection number declared
% by \cxxsec. For example:
%
%  \cxxref{basic.def.odr}
%
% Expands to the string 3.2.
\newcommand{\cxxref}[1]{%
  \csname #1 \endcsname%
}

\cxxsec{basic}{6}
\cxxsec{basic.def.odr}{6.2}
% paragraphs (based in N4741, April 2018 working draft)
\cxxsec{basic.def.odr.odrparagraph1}{12.2.2}
\cxxsec{basic.def.odr.odrparagraph2}{12.2.3}

\cxxsec{basic.scope.block}{6.3.3}
\cxxsec{basic.scope.param}{6.3.4}
\cxxsec{basic.scope.temp}{6.3.9}
\cxxsec{basic.lookup}{6.4}
\cxxsec{basic.lookup.unqual}{6.4.1}
\cxxsec{basic.lookup.qual}{6.4.3}
\cxxsec{basic.fundamental}{6.7.1}

\cxxsec{expr}{8}
\cxxsec{expr.context}{8.2.3}
\cxxsec{expr.prim.paren}{8.4.3}
\cxxsec{expr.call}{8.5.1.2}
\cxxsec{expr.type.conv}{8.5.1.3}
\cxxsec{expr.typeid}{8.5.1.8}
\cxxsec{expr.cond}{8.5.16}
\cxxsec{expr.comma}{8.5.19}
\cxxsec{expr.const}{8.6}
\cxxsec{stmt.expr}{9.2}
\cxxsec{stmt.return}{9.6.3}


\cxxsec{dcl.inline}{10.1.6}
\cxxsec{dcl.type.simple}{10.1.7.2}
\cxxsec{tref:simple.type.specifiers}{11}
\cxxsec{dcl.enum}{10.2}
\cxxsec{namespace.udecl}{10.3.3}

\cxxsec{dcl.ambig.res}{11.2}
\cxxsec{dcl.fct}{11.3.5}

\cxxsec{class}{12}
\cxxsec{class.nest}{12.2.5}
\cxxsec{class.access.base}{14.2}
\cxxsec{special}{15}
\cxxsec{class.conv.fct}{15.3.2}
\cxxsec{over.call.object}{16.3.1.1.2}
\cxxsec{over.oper}{16.5}
\cxxsec{temp.concept}{17.6.8}
\cxxsec{temp.deduct}{17.9.2}
\cxxsec{temp.dep.expr}{17.7.2.2}
\cxxsec{cpp.predefined}{19.8}
\cxxsec{meta.rqmts}{23.15.1}
